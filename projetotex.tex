\documentclass[12pt]{article}
\usepackage{fontenc}
\usepackage[utf8]{inputenc}
\usepackage[brazilian]{babel}
\usepackage{amsmath, amsfonts, amssymb}
\usepackage[top=2cm, bottom=2cm, left=2.5cm, right=2.5cm]{geometry}


\title{\textbf{Exercícios de fixação - Curso de LaTeX}}
\author{Autor: Rafael Gomes da Silva}
\date{Data: 22 de Julho de 2023}

\begin{document}
\maketitle

Abaixo seguem os exercícios resolvidos:

\begin{equation}
	\sqrt[3]{\left(\frac{2^{3}+2^{5}}{10}\right)}
\end{equation}

\begin{equation}
	\overline{(x \cdot y)^{4}} = \overline{x^{4}} \cdot \overline{y^{4}}
\end{equation}

\begin{equation}
	\frac{a}{\sin \widehat{A}} = \frac{b}{\sin \widehat{B}} = \frac{c}{\sin \widehat{C}} = 2r
\end{equation}

\begin{equation}
	\Vert \vec{u} \times \vec{v} \Vert= \Vert \vec{u} \Vert \cdot \Vert \vec{v} \Vert \cdot \sin(\theta)
\end{equation}
   
\begin{equation}
	\frac{1}{\left(\frac{2}{3} \, cm/s \right)^2} \frac{\partial^{2}\psi}{\partial \, t^2} - \frac{\partial^{2}\psi}{\partial \, x^2} = 0
\end{equation}
   
\begin{equation}
	\left(\frac{a}{b+c}\right)
\end{equation}
 
\begin{equation}
	\left(a \times b \right) + c - \left(\frac{d}{e}\right)
\end{equation}
 
\begin{equation}
	a^{\frac{m}{n}} = \sqrt[m]{n}
\end{equation}
 
\begin{equation}
	\log_3{\sqrt[3]{3}} = x
\end{equation}
 
\begin{equation}
	a = - \frac{\pi}{12} + k\frac{\pi}{2}, \; k \in \mathbb{Z}
\end{equation}

\begin{equation}
	\lim_{x \rightarrow -2} \sqrt \frac{x^{3}+2x+3}{x^{2}+5}
\end{equation}

\begin{equation}
	f(x) = \left\{
	\begin{array}{cc}
		6x-1, \, & x \neq 2 \\
		3, \quad & x = 2
	\end{array}
	\right.
\end{equation}

\begin{equation}
	\int \left(\frac{2}{\sqrt{1 - x^2} - \frac{1}{\sqrt[4]{x}}}\right)dx
\end{equation}

\begin{equation}
	\mathcal{A} = \int_{-\frac{\pi}{2}}^{\frac{\pi}{5}} cos \, \theta \, d\theta
\end{equation}

\begin{equation}
	\vec{v}= \lambda \, \vec{f}
\end{equation}

\begin{equation}
	\vec{F} = m \, \vec{a}
\end{equation}

\begin{equation}
	M_{3 \times 4}= \left(
	\begin{array}{lrlr}
		1 & 2 & 3 & 4 \\
		5 & 6 & 7 & 8 \\
		9 & 10& 11&12 \\
	\end{array}\right)
\end{equation}

\begin{equation}
	M_{2 \times 2}= \left(
	\begin{array}{lrlr}
		x + y & t - z \\
		2x - y & t+z \\
	\end{array}\right)
\end{equation}

\begin{equation}
	a_{ij} = \left\{
	\begin{array}{cc}
		2^{i + j}, & i < j \\
		i^2 + 5, & i \geq j
	\end{array}\right.
\end{equation}



\end{document}